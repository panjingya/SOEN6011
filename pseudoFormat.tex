\documentclass{article}
\usepackage{algorithm}
\usepackage{algpseudocode}

\begin{document}

\begin{algorithm}

\caption{Calculate Logarithm Function}

\textbf{Require:}  value: $x > 0$ And base: $b \neq 1 \vee b > 0$ \Comment{where $x,b \in \mathcal{R}^+$}\\
\textbf{Ensure:} $result = \log_b x$
\begin{algorithmic}[1]

\Procedure {CalculatePower}{$base$, $exponent$}
    \State $power \leftarrow 1$
    
    \For {$i \leftarrow 1, exponent$}
    \State $power \leftarrow power * base$
    \EndFor
    \State \textbf{return} $power$\Comment{It returns the base to the power exponent}
    \EndProcedure
\Statex

\Procedure {CalculateNaturalLog}{$value$}
    \State $sum \leftarrow 0$
    \State $j \leftarrow (value-1) / (value+1)$
    \For {$i \leftarrow 1, \infty$}
    \State $k \leftarrow (2 * i) -1$
    \State $sum \leftarrow sum + (1/k) * \Call{CalculatePower}{j,k} $
    \EndFor
    \State \textbf{return} $2*sum$\Comment{It returns ln using series expansion.}
    \EndProcedure
\Statex

\State $ a \leftarrow \Call{CalculateNaturalLog}{x}$\Comment{Calculates $ln x$}
\State $ b \leftarrow \Call{CalculateNaturalLog}{b}$\Comment{Calculates $ln b$}
\State $result \leftarrow a/b $\Comment{Final result of $log_b x$}

\end{algorithmic}
\end{algorithm}
\end{document}