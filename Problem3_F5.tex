\documentclass[a4paper, 11pt]{article}
\usepackage{comment} % enables the use of multi-line comments (\ifx \fi) 
\usepackage{fullpage} % changes the margin
\usepackage{hyperref}
\usepackage{amsmath}

\usepackage{booktabs} % For formal tables


\usepackage{algorithm}
\usepackage{algpseudocode}

\begin{document}
\noindent
\large\textbf{PROBLEM 3--PSEUDOCODE FORMAT}\hfill \textbf{Jingya Pan} \\
\normalsize SOEN 6011 \hfill \textbf{40044079} \\
 SOFTWARE ENGINEERING PROCESSES \hfill Due Date: 7/19/2019 \\
\hfill Github address : https://github.com/panjingya/SOEN6011.git

\section{Gamma function with Lanczos}
\begin{algorithm}

\caption{Gamma function with Lanczos approximation}

\textbf{Require:}  value: $x \in $ non-negative integer \Comment{else raise an exception}\\
\textbf{Ensure:} g is a constant that chosen arbitrarily subject to the restriction that Re(z) $>\frac{1}{2}$.
\begin{algorithmic}[1]

\Procedure {lanczosGamma}{$double $ x}
    \State if x $< \frac{1}{2}$ then
    \State \indent $\Gamma (x)={\pi  \over \sin \pi x \Gamma (1-x)}.$  \Comment{Reflection formula}
    \State g $\leftarrow$  a small integer
    \State A $\leftarrow$ a convergent with 5-10 terms \Comment{single or double floating-point precision.}
    
    
    \For {$k \leftarrow 1, A.length $}
    \State $A_{g}(x)=c_{0}+{\frac  {c_{k}}{x+k}}$
    \EndFor
    \State ${\displaystyle \Gamma (x+1)={\sqrt {2\pi }}{\left(x+g+{\tfrac {1}{2}}\right)}^{x+{\frac {1}{2}}}e^{-\left(x+g+{\frac {1}{2}}\right)}A_{g}(x)}$
    
    \State \textbf{return} $\Gamma (x)$
    \EndProcedure
\Statex

\State $result \leftarrow \Gamma (x)$ 

\end{algorithmic}
\end{algorithm}

Lanczos approximation uses the  reflection formula to make an extension to the factorial method which will be used to calculate the input smaller that $\frac{1}{2}$. For other values, it will use another formula. First of all, 5 to 10 terms of the series in an aggregation are needed to compute the gamma function with typical single or double floating-point precision and choose a fixed constant g, which will be all used to calculate the coefficients.And then bring it in the formula ${\displaystyle \Gamma (x+1)={\sqrt {2\pi }}{\left(x+g+{\tfrac {1}{2}}\right)}^{x+{\frac {1}{2}}}e^{-\left(x+g+{\frac {1}{2}}\right)}A_{g}(x)}$.

\section{Gamma function with Stirling}
\begin{algorithm}

\caption{Gamma function with Stirling's approximation}

\textbf{Require:}  value: $x \in $ non-negative integer \Comment{else raise an exception}\\
\textbf{Ensure:} x is large in absolute value
\begin{algorithmic}[1]

\Procedure {stirlingGamma}{$double $ x}
    \State \indent $\Gamma (x) = {\sqrt {\frac{2\pi}{n}}{(\frac{n}{e})}^n(1+O(\frac{1}{x}))}.$  \Comment{formula is an asymptotic expansion}
    
    \State \textbf{return} $\Gamma (x)$
    \EndProcedure
\Statex

\State $result \leftarrow \Gamma (x)$ 
\end{algorithmic}
\end{algorithm}

For all positive integers, ${\displaystyle x!=\Gamma (x+1)}$ will be applicable. However  the gamma function, unlike the factorial, is more broadly defined for all complex numbers other than non-positive integers; nevertheless, Stirling's formula may still be applied. If $Re(x) > 0$\\ then, ${\displaystyle \ln \Gamma (x)=x\ln x-x+{\tfrac {1}{2}}\ln {\frac {2\pi }{x}}+\int _{0}^{\infty }{\frac {2\arctan \left({\frac {t}{x}}\right)}{e^{2\pi t}-1}}\,{\rm {d}}t.}$, repeated integration by parts gives\\${\displaystyle \ln \Gamma (x)\sim x\ln x-x+{\tfrac {1}{2}}\ln {\frac {2\pi }{x}}+\sum _{n=1}^{N-1}{\frac {B_{2n}}{2n(2n-1)x^{2n-1}}}}$,the formula is valid for x large enough in absolute value, so the approximation method will be deduct to ${\displaystyle \Gamma (z)={\sqrt {\frac {2\pi }{z}}}\,{\left({\frac {z}{e}}\right)}^{z}\left(1+O\left({\frac {1}{z}}\right)\right).}$

\section{Advantages And Disadvantages}
\subsection{ The Lanczos approximation method}
\indent \indent Efficiency: The Lanczos approximation method evidently improves the efficiency than the original gamma function defined straightforward as ${\displaystyle \Gamma (x)=\int _{0}^{\infty }t^{x-1}e^{-t}\,dt}$ which need the integral calculating. \\

Application field: The Lanczos approximation method for gamma function has a more broad application than the normal factorial function which can only deal with the positive integer. With the formula deduction, the method is valid for arguments in the right complex half-plane, however with the  reflection formula, it can be extended to the entire complex plane without the negative integer.\\

Accuracy: As the Lanczos method is still an approximation, the result has some deviation. Some sample difference will show below in 3.3.

\subsection{ The Stirling's approximation method}
\indent\indent Accuracy: Stirling's approximation leads to accurate results for some large input, but for some relative some input, the result is not that accurate, which means the relative error occurs.\\

Efficiency: The Stirling's approximation method also improves the efficiency than the original gamma function calculating because the integral calculation really takes time \\

Application field: The Stirling's approximation for gamma function, unlike the factorial, is more broadly defined for all complex numbers other than non-positive integers; nevertheless, Stirling's approximation method may still be applied.\\

Simplicity: Comparing to the other methods implementing the gamma function, the Stirling's approximation method is relatively simple.

\subsection{The result comparation}
Gamma \hspace{3cm}		Stirling \hspace{3cm}	Lanczos\\
1.0	\hspace{3cm}	0.9221370088957891\hspace{1cm}	0.9999999999999998\\
2.0	\hspace{3cm}	0.9595021757444916\hspace{1cm}	1.0000000000000002\\
3.0	\hspace{3cm}	1.945403197115288 \hspace{1cm}	2.000000000000001\\
4.0	\hspace{3cm}	5.876543783223323 \hspace{1cm}	6.000000000000007\\
5.0	\hspace{3cm}	23.60383359151802 \hspace{1cm}	23.999999999999996
\section{References}
PUGH, G. R. (2004, November). AN ANALYSIS OF THE LANCZOS GAMMA APPROXIMATION. Retrieved from http://laplace.physics.ubc.ca/People/matt/Doc/ThesesOthers/Phd/pugh.pdf \\

\noindent
Stirling's approximation. (2019, May 18). Retrieved from https://en.wikipedia.org/wiki/Stirling's-approximation-An-alternative-derivation \\

\noindent
Lanczos approximation.(2019, June 25). Retrieved from https://en.wikipedia.org/wiki/Lanczos-approximation
\end{document}